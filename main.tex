\documentclass[12pt]{article}
\usepackage[spanish]{babel}
\usepackage[utf8]{inputenc}
\usepackage[T1]{fontenc}
\usepackage{amsmath}
\usepackage{listings}
\usepackage{xcolor}
\usepackage{geometry}
\geometry{margin=1in}

\title{\textbf{LABORATORIO 1}\\Implementación de la Sucesión de Fibonacci en MIPS32}
\author{Carlos Sánchez \\ C.I. V-31030122}
\date{}

\definecolor{codegray}{gray}{0.95}

\lstdefinelanguage{MIPS}{
  morekeywords={main,li,la,syscall,move,add,addi,sub,subi,mul,div,rem,
                beq,bne,j,jr,jal,lw,sw},
  sensitive=true,
  morecomment=[l]{\#},
  morestring=[b]",
}

\lstset{
  language=MIPS,
  backgroundcolor=\color{codegray},
  basicstyle=\ttfamily\small,
  frame=single,
  breaklines=true,
  showstringspaces=false
}

\begin{document}

\maketitle

\section*{Descripción}
Este laboratorio consiste en implementar el cálculo del enésimo término de la sucesión de Fibonacci en MIPS32, utilizando dos enfoques: uno iterativo y otro recursivo. Ambos programas reciben como entrada un número entero no negativo \( n \) y devuelven como salida el valor de \( \text{fib}(n) \).

\section*{Código Iterativo}
\begin{lstlisting}
.text
.globl main

main:
    la $a0, prompt
    li $v0, 4
    syscall

    li $v0, 5
    syscall
    move $t0, $v0

    li $t1, 0
    li $t2, 1
    beq $t0, $zero, print_fib0
    li $t3, 1
    beq $t0, $t3, print_fib1

    li $t3, 2
loop:
    add $t4, $t1, $t2
    move $t1, $t2
    move $t2, $t4
    addi $t3, $t3, 1
    ble $t3, $t0, loop

    move $a0, $t2
    li $v0, 1
    syscall
    j exit

print_fib0:
    li $a0, 0
    li $v0, 1
    syscall
    j exit

print_fib1:
    li $a0, 1
    li $v0, 1
    syscall

exit:
    li $v0, 10
    syscall

.data
prompt: .asciiz "Ingresa un numero entero no negativo: \n"
\end{lstlisting}

\section*{Código Recursivo}
\begin{lstlisting}
.text
.globl main

main:
    la $a0, prompt
    li $v0, 4
    syscall

    li $v0, 5
    syscall
    move $a0, $v0

    jal vfib

    move $a0, $v0
    li $v0, 1
    syscall

    li $v0, 10
    syscall

vfib:
    addi $t0, $zero, 1
    beq $a0, $zero, fib0
    beq $a0, $t0, fib1
    j fib

fib0:
    li $v0, 0
    jr $ra

fib1:
    li $v0, 1
    jr $ra

fib:
    addi $sp, $sp, -16
    sw $ra, 0($sp)
    sw $a0, 4($sp)

    addi $a0, $a0, -1
    jal vfib
    sw $v0, 8($sp)

    lw $a0, 4($sp)
    addi $a0, $a0, -2
    jal vfib
    sw $v0, 12($sp)

    lw $ra, 0($sp)
    lw $t0, 8($sp)
    lw $t1, 12($sp)
    addi $sp, $sp, 16
    add $v0, $t0, $t1

    jr $ra

.data
prompt: .ascii "Ingresa un numero entero no negativo: \n"
\end{lstlisting}

\section*{Preguntas y Respuestas}

\subsection*{1. ¿Cómo se implementa la recursividad en MIPS32 y qué papel cumple la pila \texttt{\$sp}?}
La recursividad se implementa mediante llamadas a funciones con \texttt{jal}, y cada llamada guarda su contexto (como \texttt{\$ra} y \texttt{\$a0}) en la pila usando \texttt{\$sp}. Esto permite que cada llamada tenga su propio entorno de ejecución sin interferencias.

\subsection*{2. ¿Qué riesgos de desbordamiento existen y cómo mitigarlos?}
Existen dos riesgos:
\begin{itemize}
  \item \textbf{Desbordamiento de pila:} muchas llamadas recursivas pueden agotar la memoria de pila.
  \item \textbf{Desbordamiento de enteros:} los valores de Fibonacci crecen rápidamente y pueden exceder los 32 bits.
\end{itemize}
Se mitigan limitando el valor de entrada y prefiriendo la versión iterativa.

\subsection*{3. Diferencias entre implementación iterativa y recursiva}
\begin{itemize}
  \item La versión iterativa usa menos memoria y es más rápida.
  \item La versión recursiva es más elegante pero consume más pila y es más lenta.
\end{itemize}

\subsection*{4. Tutorial paso a paso en MARS}
\begin{enumerate}
  \item Abrir MARS y cargar el archivo \texttt{.asm}.
  \item Ensamblar con \texttt{Assemble}.
  \item Usar \texttt{Step} para ejecutar instrucción por instrucción.
  \item Observar los registros y la pila en cada paso.
\end{enumerate}

\subsection*{5. Justificación del enfoque}
El enfoque iterativo es más eficiente y seguro en MIPS32, ya que evita el uso excesivo de la pila y es más directo. La recursividad es útil para aprendizaje, pero menos práctica en arquitecturas con recursos limitados.

\end{document}

